\documentclass[a4paper,USenglish,cleveref, autoref, thm-restate]{lipics-v2021}
%This is a template for producing LIPIcs articles. 
%See lipics-v2021-authors-guidelines.pdf for further information.
%for A4 paper format use option "a4paper", for US-letter use option "letterpaper"
%for british hyphenation rules use option "UKenglish", for american hyphenation rules use option "USenglish"
%for section-numbered lemmas etc., use "numberwithinsect"
%for enabling cleveref support, use "cleveref"
%for enabling autoref support, use "autoref"
%for anonymousing the authors (e.g. for double-blind review), add "anonymous"
%for enabling thm-restate support, use "thm-restate"
%for enabling a two-column layout for the author/affilation part (only applicable for > 6 authors), use "authorcolumns"
%for producing a PDF according the PDF/A standard, add "pdfa"

\pdfoutput=1 %uncomment to ensure pdflatex processing (mandatatory e.g. to submit to arXiv)
%\hideLIPIcs  %uncomment to remove references to LIPIcs series (logo, DOI, ...), e.g. when preparing a pre-final version to be uploaded to arXiv or another public repository

%\graphicspath{{./graphics/}}%helpful if your graphic files are in another directory

\bibliographystyle{plainurl}% the mandatory bibstyle

\title{Primitive Recursion \\ in Bicartesian Closed Categories} %TODO Please add

%\titlerunning{Dummy short title} %TODO optional, please use if title is longer than one line

\author{Jane {Open Access}}{Dummy University Computing Laboratory, [optional: Address], Country \and My second affiliation, Country \and \url{http://www.myhomepage.edu} }{johnqpublic@dummyuni.org}{https://orcid.org/0000-0002-1825-0097}{}%TODO mandatory, please use full name; only 1 author per \author macro; first two parameters are mandatory, other parameters can be empty. Please provide at least the name of the affiliation and the country. The full address is optional. Use additional curly braces to indicate the correct name splitting when the last name consists of multiple name parts.

\author{Joan R. Public\footnote{Optional footnote, e.g. to mark corresponding author}}{Department of Informatics, Dummy College, [optional: Address], Country}{joanrpublic@dummycollege.org}{[orcid]}{}

\authorrunning{J. Open Access and J.\,R. Public} %TODO mandatory. First: Use abbreviated first/middle names. Second (only in severe cases): Use first author plus 'et al.'

\Copyright{Jane Open Access and Joan R. Public} %TODO mandatory, please use full first names. LIPIcs license is "CC-BY";  http://creativecommons.org/licenses/by/3.0/

\ccsdesc[100]{\textcolor{red}{Replace ccsdesc macro with valid one}} %TODO mandatory: Please choose ACM 2012 classifications from https://dl.acm.org/ccs/ccs_flat.cfm 

\keywords{Dummy keyword} %TODO mandatory; please add comma-separated list of keywords

% \category{} %optional, e.g. invited paper

% \relatedversion{} %optional, e.g. full version hosted on arXiv, HAL, or other respository/website
%\relatedversiondetails[linktext={opt. text shown instead of the URL}, cite=DBLP:books/mk/GrayR93]{Classification (e.g. Full Version, Extended Version, Previous Version}{URL to related version} %linktext and cite are optional

%\supplement{}%optional, e.g. related research data, source code, ... hosted on a repository like zenodo, figshare, GitHub, ...
%\supplementdetails[linktext={opt. text shown instead of the URL}, cite=DBLP:books/mk/GrayR93, subcategory={Description, Subcategory}, swhid={Software Heritage Identifier}]{General Classification (e.g. Software, Dataset, Model, ...)}{URL to related version} %linktext, cite, and subcategory are optional

%\funding{(Optional) general funding statement \dots}%optional, to capture a funding statement, which applies to all authors. Please enter author specific funding statements as fifth argument of the \author macro.

% \acknowledgements{I want to thank \dots}%optional

%\nolinenumbers %uncomment to disable line numbering



%Editor-only macros:: begin (do not touch as author)%%%%%%%%%%%%%%%%%%%%%%%%%%%%%%%%%%
\EventEditors{John Q. Open and Joan R. Access}
\EventNoEds{2}
\EventLongTitle{42nd Conference on Very Important Topics (CVIT 2016)}
\EventShortTitle{CVIT 2016}
\EventAcronym{CVIT}
\EventYear{2016}
\EventDate{December 24--27, 2016}
\EventLocation{Little Whinging, United Kingdom}
\EventLogo{}
\SeriesVolume{42}
\ArticleNo{23}
%%%%%%%%%%%%%%%%%%%%%%%%%%%%%%%%%%%%%%%%%%%%%%%%%%%%%%

\usepackage{newunicodechar}
\newunicodechar{λ}{\ensuremath{\mathnormal\lambda}}
\newunicodechar{σ}{\ensuremath{\mathnormal\sigma}}
\newunicodechar{π}{\ensuremath{\mathnormal\pi}}
\newunicodechar{ℕ}{\ensuremath{\mathbb{N}}}
\newunicodechar{∷}{\ensuremath{::}}
\newunicodechar{≡}{\ensuremath{\equiv}}
\newunicodechar{∀}{\ensuremath{\forall}}
\newunicodechar{ᴸ}{\ensuremath{^L}}
\newunicodechar{ᴿ}{\ensuremath{^R}}
\newunicodechar{ʳ}{\ensuremath{^r}}
\newunicodechar{ⱽ}{\ensuremath{^V}}
\newunicodechar{⟧}{\ensuremath{\rrbracket}}
\newunicodechar{⟦}{\ensuremath{\llbracket}}
\newunicodechar{⊤}{\ensuremath{\top}}
\newunicodechar{₁}{\ensuremath{_1}}
\newunicodechar{₂}{\ensuremath{_2}}
\newunicodechar{∈}{\ensuremath{\in}}
\newunicodechar{₀}{\ensuremath{_0}}
\newunicodechar{′}{\ensuremath{'}}
\newunicodechar{ᴬ}{\ensuremath{^A}}
\newunicodechar{∘}{\ensuremath{\circ}}
\newunicodechar{𝟙}{\ensuremath{\mathbb{I}}}
\newunicodechar{ᴾ}{\ensuremath{^P}}
\newunicodechar{ᵀ}{\ensuremath{^T}}
\newunicodechar{⊎}{\ensuremath{\uplus}}
\newunicodechar{ι}{\ensuremath{\iota}}
\newunicodechar{⇐}{\ensuremath{\Leftarrow}}
\newunicodechar{∎}{\ensuremath{\mathnormal\blacksquare}}

\newcommand{\ANat}{ℕ}
\newcommand{\ATY}{\AgdaDatatype{TY}}
\newcommand{\ATy}[1]{\AgdaDatatype{Ty}\AgdaSpace{}\AgdaGeneralizable{#1}}
\newcommand{\AAlg}[1]{\AgdaDatatype{Alg}\AgdaSpace{}\AgdaGeneralizable{#1}}
\newcommand{\AFin}[1]{\AgdaDatatype{Fin}\AgdaSpace{}\AgdaGeneralizable{#1}}
\newcommand{\AList}[1]{\AgdaDatatype{List}\AgdaSpace{}\AgdaGeneralizable{#1}}
\newcommand{\AMaybe}[1]{\AgdaDatatype{Maybe}\AgdaSpace{}\AgdaGeneralizable{#1}}
\newcommand{\APR}[1]{\AgdaDatatype{PR}\AgdaSpace{}\AgdaGeneralizable{#1}}
\newcommand{\ATerm}[1]{\AgdaDatatype{Term}\AgdaSpace{}\AgdaGeneralizable{#1}}
\newcommand{\ATerms}[1]{\AgdaDatatype{Term*}\AgdaSpace{}\AgdaGeneralizable{#1}}
\newcommand{\ARanked}{\AgdaDatatype{Ranked}}
\newcommand{\ASet}{\AgdaDatatype{Set}}
\newcommand{\AVec}[2]{\AgdaDatatype{Vec}\AgdaSpace{}\AgdaGeneralizable{#1}\AgdaSpace{}\AgdaGeneralizable{#2}}
\newcommand{\AExp}[2]{\AgdaDatatype{Exp}\AgdaSpace{}\AgdaGeneralizable{#1}\AgdaSpace{}\AgdaGeneralizable{#2}}
\newcommand{\AHVec}[2]{\AgdaDatatype{HVec}\AgdaSpace{}\AgdaGeneralizable{#1}\AgdaSpace{}\AgdaGeneralizable{#2}}
\newcommand{\ATop}{\AgdaDatatype{\ensuremath\top}}
\newcommand{\Afold}{\AgdaInductiveConstructor{fold}}
\newcommand{\Aind}{\AgdaInductiveConstructor{ind}}
\newcommand{\Acon}{\AgdaInductiveConstructor{con}}
\newcommand{\AP}{\AgdaInductiveConstructor{P}}
\newcommand{\Asymbols}[1]{\AgdaField{symbols}\AgdaSpace{}\AgdaBound{#1}}
\newcommand{\Arank}[1]{\AgdaField{rank}\AgdaSpace{}\AgdaBound{#1}}
\newcommand{\Asin}[1]{\AgdaField{sin*}\AgdaSpace{}\AgdaBound{#1}}
\newcommand{\Asout}[1]{\AgdaField{sout}\AgdaSpace{}\AgdaBound{#1}}
\newcommand{\Anil}{\AgdaInductiveConstructor{[]}}
\newcommand{\Aone}[1]{\AgdaInductiveConstructor{[}\AgdaSpace{}#1\AgdaSpace{}\AgdaInductiveConstructor{]}}
\newcommand{\Atwo}[2]{\AgdaInductiveConstructor{[}\AgdaSpace{}%
  #1\AgdaSpace{}\AgdaInductiveConstructor{,}\AgdaSpace{}%
  #2\AgdaSpace{}\AgdaInductiveConstructor{]}}
\newcommand{\Asubnull}{\AgdaFunction{sub₀}}
\newcommand{\Asubst}{\AgdaFunction{subst}}
\newcommand{\ASub}{\AgdaFunction{Sub}}
\newcommand{\Afmap}{\AgdaFunction{fmap}}
\newcommand{\AGNat}{\AgdaFunction{G-Nat}}
\newcommand{\Aeq}{\AgdaFunction{eq-unfold}}
\newcommand{\Aext}{\AgdaFunction{ext}}
\newcommand{\ACZ}{\AgdaInductiveConstructor{Z}}
\newcommand{\ACS}{\AgdaInductiveConstructor{σ}}
\newcommand{\ATZ}{\AgdaInductiveConstructor{`𝟘}}

%%% Local Variables:
%%% mode: latex
%%% TeX-master: "jfpmain"
%%% End:


\begin{document}
\newcommand\Nat{\ensuremath{\mathbb{N}}}
\newcommand{\many}[2]{{#1}_0,\dots,{#1}_{#2-1}}
\newcommand{\xs}{\many{x}}
\newcommand{\xn}{\xs{n}}

\maketitle

%TODO mandatory: add short abstract of the document
\begin{abstract}
  The standard definition of PR-NAT, the system of primitive recursive functions on natural
  numbers, is given in a pointfree style. Generalizations thereof like
  System T or systems with inductive types are mostly embedded in
  typed lambda calculi.

  We present two pointfree languages that generalize primitive
  recursion on natural numbers. PR1-IND defines
  first-order primitive recursive functions on inductive types in a
  distributive
  category. PR-IND extends PR1-IND to higher-order functions in a
  bicartesian closed category.

  We give the first complete, mechanized semantic description of
  PR1-IND and PR-IND in Agda, embeddings from PR-NAT to PR1-IND and
  PR1-IND to PR-IND, and an elementary proof of the folklore theorem
  that every bicartesian closed category is distributive. We cannot
  find a similar elementary proof in published literature.
\end{abstract}

\section{Introduction}
\label{sec:introduction}

Primitive recursive functions, PR-NAT, are a class of total computable
functions on natural numbers. They are a standard topic in any
class on computability theory and serve as a stepping stone for
defining general recursive functions
\cite{kleene36:_gener_recur_funct_natur_number}.
System~T adds higher-order functions to PR-NAT (e.g.,
\cite[Chapter 9]{DBLP:books/cu/Ha2016}). The resulting class of functions is a
proper extension of PR-NAT. For example, the
Ackermann function \cite{szasz1991machine}, Knuth's uparrow
notation, and Conway's arrow notation are total functions on natural
numbers, which are not primitive recursive, but definable in System~T
\cite{DBLP:journals/corr/Widemann16}.

The concept of primitive recursion can be generalized in several
dimensions. One dimension is first-order vs.\ higher-order,
corresponding to PR-NAT vs.\ System~T. A second dimension generalizes
to functions over words, terms, and many-sorted algebras. A third
dimension generalizes to vector-valued functions aka LOOP-computable
functions. The second and third dimensions can be subsumed by
situating primitive recursion in a bicartesian category, i.e., a
first-order language with finite product and sum types augmented with
an inductive type, which we call PR1-IND.  Closely related, but
subtly different, languages are discussed in Harper's book
\cite[Chapter 15]{DBLP:books/cu/Ha2016} and in the bananas paper
\cite{DBLP:conf/fpca/MeijerFP91}. 

Adding the higher-order dimension to PR1-IND leads to studying
primitive recursion in a bicartesian closed category, as considered
by Meijer and Hutton \cite{DBLP:conf/fpca/MeijerH95} and which is
implicitly and often unknowingly used by Haskell programmers. Such
categories have all finite products and coproducts (sums) as well as
exponentials. We call this language PR-IND.

We start our work from the observation that the standard definition of
PR-NAT is very close to a categorical description. Taken literally,
primitive recursive functions are defined in a pointfree style by
composing and combining predefined arrows. We define syntax and
semantics of PR-Nat in Agda, give examples, and prove their correctness.

Inspired by this observation, we go on to define PR1-IND in the same
pointfree style. Defining the semantics turns out to be surprisingly
subtle because, unlike Harper's definition \cite[Chapter
15]{DBLP:books/cu/Ha2016}, PR1-IND admits nested inductive types.
Clearly, it should be possible to translate any PR-NAT program into a
semantically equivalent program in PR1-IND, but this attempt reveals
that a bicartesian category is not sufficient for this translation,
but a distributive category is needed. That is, distributivity needs
to be added to our language as a family of axiomatic arrows.

Finally, we extend PR1-IND with exponentials to PR-IND. The definition
of the semantics smoothly extends the one for PR1-IND. As an
additional twist, surprising to the authors who are not category
theorists, the distributivity axiom becomes obsolete as it derivable in
a bicartesian closed category. We give an elementary proof of this
folklore result \cite{https://doi.org/10.48550/arxiv.1406.0961}. 

All definitions, results, and proofs are mechanized using Agda and
publicly available in a GitHub
repository.\footnote{\url{https://github.com/proglang/PrimRec/}
  release 2023-02 corresponds to the submitted paper} 

\section{Primitive recursion on natural numbers}
\label{sec:prim-recurs-natur}

Here is a typical textbook definition~\cite{martin2019logik,wiki:Primitive_recursive_function}:


The set of primitive recursive functions is the smallest family PR of functions
from ${ℕ}^n$  to ${ℕ}$, for $n\in {ℕ}$, that contains the following
basic functions:
\begin{itemize}
\item for all $n\in {ℕ}$ the $n$-ary constant zero function
  \begin{align*}
	Z^n  &\colon \mathbb{N}^n \rightarrow \mathbb{N}  &
	Z^n    (\xn) &= 0  
  \end{align*}
\item the successor function
  \begin{align*}
	S &\colon \mathbb{N} \rightarrow \mathbb{N} &
	S  (x)  &=  x + 1
  \end{align*}
\item for all $n\in {ℕ}$ and $0\le i< n$ the $n$-ary projection on
  the $i$-th argument
\begin{align*}
	\pi^{n}_{i} &\colon \mathbb{N}^n \rightarrow \mathbb{N} &
	\pi^{n}_{i}  (x_0,\dots,x_i,\dots,x_{n-1})  &= x_i
\end{align*}
\end{itemize}
PR is closed under the operations \emph{composition} and \emph{primitive recursion}.
\begin{description}
\item[Composition] 
If $f$ is a primitive recursive function with arity $m \in {ℕ}$ and
$\many{g}{m}$ are primitive recursive functions with arity $n$ then
the composition of $f$ and $\many gm$ is an $n$-ary primitive recursive function
$h := C (h,(\many gm)) \colon \mathbb{N}^n \rightarrow \mathbb{N} $ defined by
\begin{align*}
	h  (\xn)  &= f (g_1(\xn),\dots,g_m(\xn)) \\
\end{align*}

\item[Primitive recursion] 
If $g$ and $h$ are primitive recursive functions with arities $n$ and
$n+2$ then $f := P (g,h)$ is a primitive recursive function of arity $n+1$ that is defined as follows.
\begin{align*}
	f&\colon \mathbb{N}^{n+1} \rightarrow \mathbb{N} \\
	f  (0,\xn)  &= g (\xn) \\
	f  (y + 1,\xn) &= h(  f(y,\xn),y,\xn)
\end{align*}

\end{description}

Clearly, PR is inductively defined, but unfortunately the boundary
between syntax and semantics is blurred.


In programming languages terminology, primitive
recursive functions are given by a domain-specific language where a
sentence of the language specifies a function of type ${ℕ}^n \to {ℕ}$
in a pointfree style. The syntax of the language is inductively
defined and consequently its semantics is defined by induction on the
syntax, thus cleanly separating syntax and semantics.


\section{Frontier}



, it is less well known that primitive recursive
functions generalize in several dimensions. One dimension generalizes
to words, terms, and many-sorted algebras. Another dimension
generalizes to vector-valued functions aka LOOP-computable
functions. Both of these dimensions are 
subsumed by first-order primitive recursive functions on inductive
types in a bicartesion category (PR1-IND). 
A third dimension adds higher-order functions to obtain System-T from
primitive recursion on natural numbers; analogous systems for
words, terms, and many-sorted algebras are conceivable. At this point, we can subsume
all previous systems in a simply-typed lambda calculus with
inductive types and a primitive recursor (PR-IND). This calculus is
situated in a bicartesian closed category.

We exhibit executable definitions in Agda of several classes of primitive
recursive functions in pointfree style and give embeddings from each
class into the next more general class. We particularly emphasize
PR1-IND and PR-IND.
While the first-order language PR1-IND requires a distributivity axiom
(i.e., the underlying category is distributive), this axiom is no
longer required in PR-IND, because its underlying category has
exponentials.
This finding corresponds to a folklore theorem that every bicartesian
closed category is distributive and we give an elementary proof of
that theorem.
All definitions, embeddings, and proofs are given in Agda.


%%
%% Bibliography
%%

%% Please use bibtex, 

\bibliography{jfprefs}

\end{document}
