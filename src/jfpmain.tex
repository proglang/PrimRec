\documentclass{jfp}
% "nolinenum" class option is used to disable the line numbers.

\usepackage{graphicx}
\usepackage{amssymb}
 \usepackage{amsmath}
\usepackage{amsfonts}
\usepackage{stmaryrd}
\usepackage{latex/agda}
%\usepackage{biblatex}
% \usepackage{unicode-math}
% \setmathfont{XITS Math}
% \setmathfont{XITSMath-Regular}
% [    Extension = .otf,
%       BoldFont = XITSMath-Bold,
% ]

\usepackage{newunicodechar}
\newunicodechar{λ}{\ensuremath{\mathnormal\lambda}}
\newunicodechar{σ}{\ensuremath{\mathnormal\sigma}}
\newunicodechar{π}{\ensuremath{\mathnormal\pi}}
\newunicodechar{ℕ}{\ensuremath{\mathbb{N}}}
\newunicodechar{∷}{\ensuremath{::}}
\newunicodechar{≡}{\ensuremath{\equiv}}
\newunicodechar{∀}{\ensuremath{\forall}}
\newunicodechar{ᴸ}{\ensuremath{^L}}
\newunicodechar{ᴿ}{\ensuremath{^R}}
\newunicodechar{ʳ}{\ensuremath{^r}}
\newunicodechar{ⱽ}{\ensuremath{^V}}
\newunicodechar{⟧}{\ensuremath{\rrbracket}}
\newunicodechar{⟦}{\ensuremath{\llbracket}}
\newunicodechar{⊤}{\ensuremath{\top}}
\newunicodechar{₁}{\ensuremath{_1}}
\newunicodechar{₂}{\ensuremath{_2}}
\newunicodechar{∈}{\ensuremath{\in}}
\newunicodechar{₀}{\ensuremath{_0}}
\newunicodechar{′}{\ensuremath{'}}
\newunicodechar{ᴬ}{\ensuremath{^A}}
\newunicodechar{∘}{\ensuremath{\circ}}
\newunicodechar{𝟙}{\ensuremath{\mathbb{I}}}
\newunicodechar{ᴾ}{\ensuremath{^P}}
\newunicodechar{ᵀ}{\ensuremath{^T}}
\newunicodechar{⊎}{\ensuremath{\uplus}}
\newunicodechar{ι}{\ensuremath{\iota}}
\newcommand\Nat{ℕ}

\newcommand{\xn}{x_1,\dots,x_n }
\newcommand{\xs}[1]{x_1\dots x_{#1}}

\newcommand{\ANat}{ℕ}
\newcommand{\ATY}{\AgdaDatatype{TY}}
\newcommand{\ATy}[1]{\AgdaDatatype{Ty}\AgdaSpace{}\AgdaGeneralizable{#1}}
\newcommand{\AAlg}[1]{\AgdaDatatype{Alg}\AgdaSpace{}\AgdaGeneralizable{#1}}
\newcommand{\AFin}[1]{\AgdaDatatype{Fin}\AgdaSpace{}\AgdaGeneralizable{#1}}
\newcommand{\AList}[1]{\AgdaDatatype{List}\AgdaSpace{}\AgdaGeneralizable{#1}}
\newcommand{\AMaybe}[1]{\AgdaDatatype{Maybe}\AgdaSpace{}\AgdaGeneralizable{#1}}
\newcommand{\APR}[1]{\AgdaDatatype{PR}\AgdaSpace{}\AgdaGeneralizable{#1}}
\newcommand{\ATerm}[1]{\AgdaDatatype{Term}\AgdaSpace{}\AgdaGeneralizable{#1}}
\newcommand{\ATerms}[1]{\AgdaDatatype{Term*}\AgdaSpace{}\AgdaGeneralizable{#1}}
\newcommand{\ARanked}{\AgdaDatatype{Ranked}}
\newcommand{\ASet}{\AgdaDatatype{Set}}
\newcommand{\AVec}[2]{\AgdaDatatype{Vec}\AgdaSpace{}\AgdaGeneralizable{#1}\AgdaSpace{}\AgdaGeneralizable{#2}}
\newcommand{\AExp}[2]{\AgdaDatatype{Exp}\AgdaSpace{}\AgdaGeneralizable{#1}\AgdaSpace{}\AgdaGeneralizable{#2}}
\newcommand{\AHVec}[2]{\AgdaDatatype{HVec}\AgdaSpace{}\AgdaGeneralizable{#1}\AgdaSpace{}\AgdaGeneralizable{#2}}
\newcommand{\ATop}{\AgdaDatatype{\ensuremath\top}}
\newcommand{\Afold}{\AgdaInductiveConstructor{fold}}
\newcommand{\Aind}{\AgdaInductiveConstructor{ind}}
\newcommand{\Acon}{\AgdaInductiveConstructor{con}}
\newcommand{\AP}{\AgdaInductiveConstructor{P}}
\newcommand{\Asymbols}[1]{\AgdaField{symbols}\AgdaSpace{}\AgdaBound{#1}}
\newcommand{\Arank}[1]{\AgdaField{rank}\AgdaSpace{}\AgdaBound{#1}}
\newcommand{\Asin}[1]{\AgdaField{sin*}\AgdaSpace{}\AgdaBound{#1}}
\newcommand{\Asout}[1]{\AgdaField{sout}\AgdaSpace{}\AgdaBound{#1}}
\newcommand{\Anil}{\AgdaInductiveConstructor{[]}}
\newcommand{\Aone}[1]{\AgdaInductiveConstructor{[}\AgdaSpace{}#1\AgdaSpace{}\AgdaInductiveConstructor{]}}
\newcommand{\Atwo}[2]{\AgdaInductiveConstructor{[}\AgdaSpace{}%
  #1\AgdaSpace{}\AgdaInductiveConstructor{,}\AgdaSpace{}%
  #2\AgdaSpace{}\AgdaInductiveConstructor{]}}
\newcommand{\Asubnull}{\AgdaFunction{sub₀}}
\newcommand{\Asubst}{\AgdaFunction{subst}}
\newcommand{\ASub}{\AgdaFunction{Sub}}
\newcommand{\Afmap}{\AgdaFunction{fmap}}
\newcommand{\AGNat}{\AgdaFunction{G-Nat}}
\newcommand{\Aeq}{\AgdaFunction{eq-unfold}}
\newcommand{\Aext}{\AgdaFunction{ext}}
\newcommand{\ACZ}{\AgdaInductiveConstructor{Z}}
\newcommand{\ACS}{\AgdaInductiveConstructor{σ}}
\newcommand{\ATZ}{\AgdaInductiveConstructor{`𝟘}}

%%% Local Variables:
%%% mode: latex
%%% TeX-master: "jfpmain"
%%% End:


\newunicodechar{∎}{\ensuremath{\mathnormal\blacksquare}}


\begin{document}

\journaltitle{JFP}
\cpr{Cambridge University Press}
\doival{10.1017/xxxxx}

% \lefttitle{\LaTeX\ Supplement}
% \righttitle{Journal of Functional Programming}

\totalpg{\pageref{lastpage01}}
\jnlDoiYr{2022}

\title{Functional Pearl:\\ Variations on Primitive Recursion}

\begin{authgrp}
  \author{Maximilian Hertenstein} \qquad
  \author{Peter Thiemann}
\affiliation{University of Freiburg\\
        (\email{\{mh,thiemann\}@informatik.uni-freiburg.de})}
\end{authgrp}

%\received{20 March 1995; revised 30 September 1998}

\begin{abstract}
Primitive recursive functions are a class of computable total
functions on natural numbers. While they are a standard topic in any
class on computability, it is less well known that primitive recursive
functions generalize in several dimensions. One dimension generalizes
to words, terms, and many-sorted algebras. Another dimension
generalizes to vector-valued functions. A third dimension adds
higher-order functions to obtain System-T and (even more general) STLC
with inductive types. 

We exhibit executable definitions of several classes of primitive
recursive functions and give embeddings from each class into the more
general class. STLC with inductive types is shown to be the most
general language in this class of definitions. All definitions and
embeddings are given in Agda and proved correct.
\end{abstract}
\begin{keywords}
primitive recursion, system-T, inductive types, algebras
\end{keywords}

\maketitle[F]

\section{Introduction}
\label{sec:introduction}

Primitive recursive functions are a standard topic in any class on
computability. 
The usual treatment discusses them as total, computable functions on
natural numbers. They are shown to be equivalent to
\emph{Loop programs}  \cite{loopsRitchie}, which are simple
imperative programs with bounded iteration.
Despite their simplicity, it is possible to define the basic
arithmetic functions as well as interesting functions like
\emph{nthPrime} as primitive recursive functions.

Nevertheless,  primitive recursive functions---when restricted to
first-order functions $\mathbb{N}^n \to \mathbb{N}$---are a proper subset of 
the class of total recursive functions. For example, the
\emph{Ackermann function} \cite{szasz1991machine}, Knuth's uparrow
notation, and Conway's arrow notation yield total functions, which are
not primitive recursive.
If we allow arbitrary (higher-order)
functions over natural numbers, it is possible to define functions
like the Ackermann function using primitive recursion
\cite{DBLP:journals/corr/Widemann16}. 

% In the definition of recursive functions, there is a additional
% construct called $\mu$\emph{-Operator}, that can be used to define
% partial functions. 

% Recursive functions are as powerful as the untyped lambda calculus
% and touring-machines \cite{threeModels}
% \cite[92-94]{ziegler2017godelsche}. 
 




Here is a typical textbook definition of the set of primitive
recursive functions \cite{martin2019logik,wiki:Primitive_recursive_function}:


The set of primitive recursive functions is the smallest family $PR$ of functions
from ${ℕ}^n$  to ${ℕ}$, for $n\in {ℕ}$, that contains the following
basic functions:
\begin{itemize}
\item for all $n\in {ℕ}$ the $n$-ary constant zero function
  \begin{align*}
	Z^n  &\colon \mathbb{N}^n \rightarrow \mathbb{N}  &
	Z^n    (\xn) &= 0  
  \end{align*}
\item the successor function
  \begin{align*}
	S &\colon \mathbb{N} \rightarrow \mathbb{N} &
	S  (x)  &=  x + 1
  \end{align*}
\item for all $n\in {ℕ}$ and $1\le i\le n$ the $n$-ary projection on
  the $i$-th argument
\begin{align*}
	\pi^{n}_{i} &\colon \mathbb{N}^n \rightarrow \mathbb{N} &
	\pi^{n}_{i}  (x_1,\dots,x_i,\dots,x_n)  &= x_i
\end{align*}
\end{itemize}
PR is closed under the operations \emph{composition} and \emph{primitive recursion}.
\begin{description}
\item[Composition] 
If $f$ is a primitive recursive function with arity $m \in {ℕ}$ and $g_1,
\dots, g_m$ are primitive recursive functions with arity $n$ then the
composition of $f$ and $g_1, \dots, g_m$ is an $n$-ary primitive recursive function
$h := C (h,(g_1,\dots, g_m)) \colon \mathbb{N}^n \rightarrow \mathbb{N} $ defined by
\begin{align*}
	h  (\xn)  &= f (g_1(\xn),\dots,g_m(\xn)) \\
\end{align*}

\item[Primitive recursion] 
If $g$ and $h$ are primitive recursive functions with aritys $n$ and
$n+2$ then $f := P (g,h)$ is a primitive recursive function of arity $n+1$ that is defined as follows.
\begin{align*}
	f&\colon \mathbb{N}^{n+1} \rightarrow \mathbb{N} \\
	f  (0,\xn)  &= g (\xn) \\
	f  (y + 1,\xn) &= h(  f(y,\xn),y,\xn)
\end{align*}

\end{description}

Clearly, PR is inductively defined, but unfortunately the boundary
between syntax and semantics is blurred.


In programming languages terminology, primitive
recursive functions are given by a domain-specific language where a
sentence of the language specifies a function of type ${ℕ}^n \to {ℕ}$
in a pointfree style. The syntax of the language is inductively
defined and consequently its semantics is defined by induction on the
syntax, thus cleanly separating syntax and semantics.

The following Agda definition formalizes syntax and semantics of pr
functions precisely (and strictly separated).
\input{latex/PR-Nat}

It is a good exercise to remind ourselves that the standard arithmetic
functions are pr defineable and correspond to their definitions in
Agda. Here is the easy case for addition; the online supplement
considers further operations. 
\input{latex/PR-Nat-Example}
The function \AgdaFunction{addP} is defined by primitive recursion, which
decomposes the first argument. The $g$-function is just the identity;
the $h$-function composes the successor with inductive result.
The equivalence proof is a
straightforward induction because the structure of the definitions
matches.

There are alternative definitions of primitive recursion where the
zero function takes any number of arguments or where there are
arbitrary constants. It is easy to show that they are equivalent to
the definition that we adopt here.

\section{From numbers to words, terms, and many-sorted algebras}
\label{sec:from-numbers-words}

While primitive recursion on natural numbers is well-known, it is less
well-known that primitive recursion can be defined on more general
algebras. In particular, we consider words over an alphabet $A$, terms
over a ranked alphabet $A$, and many-sorted terms. 

\subsection{Primitive recursion on words}
\label{sec:prim-recurs-words}

The textbook definition of $A^*$, the sets of words over an alphabet $A$ goes like this:
\begin{align*}
  A^0 &= \{ \varepsilon \} & A^{n+1} &= A \times A^n & A^* &=
  \bigcup^{n \in ℕ} A^n
\end{align*}
This definition slightly obscures the underlying algebraic structure, a list of
elements of $A$, which is better described as a fixed point:
$A^*\cong \ATop + A \times A^*$ where $\ATop$ is a one-element type.
Compared with the natural numbers, we can say that a word is
constructed from one nullary constructor, typically called
$\varepsilon$, and then a successor, $\sigma_a$, for each symbol $a\in A$.

Rosza Peter~\cite{peter35:_uber_zusam_begrif_funkt} defines primitive recursion on words
accordingly. The $n$-ary $0$-function becomes the $n$-ary
$\varepsilon$-function; the single successor $\sigma$ generalizes to a
family of successors $\sigma_a$; projection and composition do not
change; and for primitive recursion we have a $g$-function to handle the
case for $\varepsilon$ and a family of functions $h_a$ to handle the
case for $\sigma_a$. The arities of the functions are the same as before.
\input{latex/PR-Words}
Natural numbers arise as the special case where the alphabet $A$ is a
one-element set: we can prove that $\Nat \cong \text{\AList\ATop}$. Hence, we
can soundly embed pr on natural numbers into pr on words over an
alphabet by instantiating $A$ to $\ATop$.
\input{latex/NatsToWords}

\subsection{Primitive recursion on terms}
\label{sec:prim-recurs-trees}

To move from words to terms requires generalizing alphabets to
\emph{ranked alphabets}. A ranked alphabet consists of a set of
symbols and a ranking function that assigns each symbol a rank (arity) in \Nat.
\input{latex/PR-Trees}
Again, we can establish pr over words as a special case of pr on
terms. Given an alphabet $A$ we define a ranked alphabet with symbols
$A + \ATop \cong \text{\AMaybe{A}}$ and ranks defined such that the element
from $\ATop$ has arity $0$ and all other element from $A$ have arity
$1$. This mapping of types also determines the embedding of values.
\input{latex/WordsToTrees}

\subsection{Primitive recursion on many-sorted terms}
\label{sec:prim-recurs-sort}

A close look at the definition of the datatype \ATerm{R} shows that
terms are akin to singly recursive algebraic datatypes. To see that,
imagine that we specialize the single constructor {\Acon} with respect
to its first argument $a\in{}$\Asymbols{R}. This specialization yields
an algebraic datatype with constructors \Acon$_a$, one for each
element of \Asymbols{R}. The arity of such a constructor is given by
the rank of $a$.

Many-sorted terms are closely related to \emph{mutually recursive} algebraic
datatypes. To define many-sorted terms, we start with a set of sorts
$S$. The cardinality of $S$ determines the number of mutually
recursive types. 
An $S$-sorted alphabet consists of a set of symbols and a sorting
function that assigns each symbol an arity of the form $(w, s)$ where
$w \in S^*$ and $s \in S$. This definition translates roughly to:
every constructor has a simple first-order type.
\input{latex/PR-HTrees}

\section{Primitive recursion for inductive types}
\label{sec:prim-recurs-gener}

\input{latex/PR-CC-ind}

To complete the picture, we connect the term universe with inductive
types generated from sums and products. The modeling of inductive
types is taken from Harper's textbook \cite{DBLP:books/cu/Ha2016}.

We start with defining the type language.
\ccDataTy
Types are indexed by a number $n \in \ANat$ denoting the number of free
type variables in the type. A type is either a unit type, a product, a
sum, a variable, or an inductive type. The body of an inductive type
binds a new type variable for recursive occurrences of the
type. Mostly, we need \ATy{0} which we abbreviate to $\ATY$.

To define primitive recursion on this type structure in pointfree
style, we have to extend our vocabulary compared to previous attempts.
\ccDataPR
The first part defines arrows concerned with categorical structure,
the injection into the unit type, the identity arrow at all
types, and type-respecting composition of arrows.\footnote{The
  cognoscenti will see that we are defining the arrows of 
  a cartesian closed category.} The second part has arrows to
introduce and eliminate product types. The third part has arrows to
introduce and eliminate sum types, which are very obviously dual to
the respective arrows involving products. The fourth and final part
defines arrow to introduce {\Afold} and eliminate inductive types
{\AP}. This treatment is still 100\% first order, without higher-order functions.

In the definition of {\Afold} and {\AP}, we find that $G :$~\ATy1, a type with a
single type variable. It will be important to view $G$ as a functor
that acts on types $T:$~{\ATY} by substitution, i.e., $T \mapsto$~\Asubnull~$T$~$G$. 

Elimination of inductive types is by primitive recursion /
paramorphism. The arrow constructed by {\AP} maps a pair of an element
of an inductive type and an element of another type $U$ to a result type
$T$. The type $U$ corresponds to the non-recursion parameters in the
traditional setting (cf.\ Section~\ref{sec:from-numbers-words}). 
The single parameter $h$ of {\AP} unifies all subordinate
arrows. Thanks to the rich type structure, the dispatch on different
``term constructors'' can be internalized in the calculus. Thus, the
input of $h$ is a pair where the top-level induction for $G$ is
unfolded and applied to a pair of $T$, the result of the inductive
evaluation, and \Aind~$G$, the inductive subterm.

To define evaluation, we first define the interpretation of types in Agda.
\ccDataAlg
We reuse as much structure as possible by mapping unit, product, and
sum to the respective Agda types. We interpret all inductive types with a
single generic inductive type \AAlg{G}, which is indexed by its generating
functor $G$.\footnote{Technically speaking, this definition uses
  induction recursion \cite{DBLP:journals/apal/DybjerS03}.}

With this machinery in place, the definition of the evaluation
function is straightforward.
\ccFunEval
The only interesting case is the one for primitive recursion. It is
interpreted by a function taking a pair of \Aind~$G$ and some $U$. The
first component of the argument must evaluate to some \Afold~$x$, so the $x$ corresponds
to the constructor arguments, and the second component $u:U$ is the
extra argument. Now we run the function $h$ essentially on $x :{}
$\Asubnull~(\Aind~$G$)~$G$ and $u$, but 
after replacing recursive occurrences of the inductive type by a pair
of the subterm and the recursive call of {\AP}~$h$ on the subterm. We
identify the recursive occurrences by traversing $x$ according to the
structure functor $G$ using the $\Afmap$ function below, which
implements the action of $G$ on functions. 
\ccFunFmap

It turns out that we can map all flavors of primitive recursion from
Section~\ref{sec:from-numbers-words} into primitive recursion in terms
of inductive types. The main difference is that in
Sections~\ref{sec:prim-recurs-words}-\ref{sec:prim-recurs-sort} we
never assumed finiteness of (ranked) alphabets. To successfully map into
inductive types, alphabets must be finite and this requirement
matches the usual mathematical definition of alphabets.

As a simple example, we define the mapping from primitive recursion on
natural numbers into inductive types. We start with the functor $G$
for natural numbers:
\ccDefGNat

The encoding is straightforward, but it reveals a shortcoming of our
combinator language for inductive types. To start with, it requires an
encoding of vectors and vector lookup.
\ccFunMkvec
\ccDefNatToInd
The encodings of the constructors apply {\Afold} to the left and right
injection, respectively. Projection and composition are as before. For
the encoding of primitive recursion, we need two helper arrows. One
arrow that implements associativity of the product and another that
implements distributivity of sum over product. While the first one is
definable with the arrows given in the definition of {\APR n}, the
distributivity arrow has to be postulated as a new primitive!
\ccFunAssocDist

\section{Vector-valued primitive recursion}
\label{sec:vect-valu-prim}

\section{System~T and beyond}
\label{sec:system-t-beyond}

Gödel's System~T comprises the simply-typed lambda calculus extended
with natural numbers and a primitive recursor. While this calculus
enjoys strong normalisation, it is nevertheless possible to encode the
Ackermann function, as well as other fast-growing functions (see
\cite{DBLP:books/cu/Ha2016,DBLP:journals/corr/Widemann16}), which are known not 
to be primitive recursive. These definitions are possible because
System~T admits primitive recursive definitions at higher-order types.

If we restrict System~T to first-order functions, then we can show
that this version is equally expressive as primitive recursive
functions on natural numbers. 

To define this restriction, we use an intrinsic representation of
lambda terms as a datatype \AExp{n}{m} where first index is the size
of the context (number of free variables) and the second one is the
number of arguments. All variables and arguments will have type \ANat.
\input{latex/System-T0}
\defSTZero

To evaluate the iterator of this language we define paramorphisms for
natural numbers \cite{DBLP:conf/fpca/MeijerFP91}. 

\input{latex/EvalPConstructor}
\para
\evalST


To prove equivalence with standard primitive recursion, we start by defining a
function that returns closed lambda terms and prove that their
semantics matches the corresponding constructors of primitive recursion.

\input{latex/PR-SystemT0-Embedding}
\prToStSig
\embedPRSTSoundSig


We give a detailed explanation for the projection and sketch the ideas for the other constructors.

The following functions and lemmas are useful in all parts of this
proof. They relate the evaluation of a term and the same term where
the outermost free variables are abstracted.

\prepLambdas

\input{latex/VecProperties}


These lemmas prove that evaluation of a closed expression generated with
\AgdaFunction{prepLambdas} corresponds to evaluation of the wrapped
expression with the reversed arguments as the context.\footnote{
The function \AgdaFunction{++r} takes two vectors, reverses the first,
and prepends it to the second. We write $^R$ for a function that
reverses a vector using \AgdaFunction{++r}.
% \appendR 
}

Getting back to the case of projection, we need to take the opposite ($n - i$) of the index of a
projection as the deBruijn index in the term used to mimic the projection.

\mkProj

\lookupOpRev

It remains to state the corresponding lines in the definition of the
translation and the proof of the embedding:
\prToStProj
\embedPRSTSoundProj

It is not difficult to define the remaining cases in Agda, so we just
give a high-level overview of the expressions we use to represent the zero functions, composition, and primitive recursion.
\begin{align*}
  \text{$n$-ary zero: } & \lambda \xs{n} \rightarrow 0 \\
  \text{composition: } & \lambda \xs{n} \rightarrow  f \ (g_1 \ \xs{n})
                        \dots (g_m \ \xs{n}) \\
  \text{primitive recursion: } & \lambda \xs{n+1} \rightarrow \mathtt{PRecTC} \ (\lambda a b . h \, a \, b \, x_2 \dots x_{n+1}) \ (g \, x_2 \dots x_{n+1}) \ x_1 
\end{align*}


It is a well-known fact that function can be expressed in System~T,
which are not primitive recursive. The Ackermann function is an
example. The key for this expressiveness lies in higher-order
types. For our first-order restriction, System~T0, we can show every
term in this language can be mapped to a primitive recursive function
with the same semantics.  

We start by defining a function that maps expressions in System~T0 to primitive recursive functions.
We cannot restrict this function to closed terms, because we need to
invoke it recursively on the body of a lambda expression, which
increases the context by a new free variable.  


So we need a more general embedding and soundness-theorem, in which we
add the number of arguments and the size of the context to get the
arity of the primitive recursive function. 

\input{latex/System-T0-PR-Embedding}

\sTtoPRSignature

\sTtoPRSoundSig

The idea behind these definitions is that every element of
\AgdaDatatype{Exp n m} can be transformed to a closed term with type
\AgdaDatatype{Exp 0 (n + m)} by abstracting \AgdaBound{n} binders. By
the lemma \AgdaFunction{abstrEval}, we know that evaluating this
expression on the reverse-concatenation of the context and the arguments
gives the same result as evaluating the inner expression with the
context and the arguments. 
Hence, \AgdaFunction{sTtoPR} maps an expression of System~T0 to 
a primitive recursive function with the same semantics after closing with  \AgdaFunction{abstr}.
%%%now about the semantic of this new term. 

When defining the function \AgdaFunction{sTtoPR} we have to consider
that in the soundness-theorem above, for an expression the
corresponding primitive recursive function 
does not get evaluated just with the arguments of this expression but with the reverse-concatenation of the context and the arguments.
%
For example, the successor function in System~T0 can be evaluated with
a context of an arbitrary size $n$ and one argument. If we map this
function to a primitive recursive function, we can evaluate it with $n
+ 1$ arguments. Only the last of these arguments is needed to compute
the result, so we have to ignore all other arguments using a suitable projection. 
\sTtoPRSuc
%That is done with a projection that returns the last argument of the input of the primitive recursive function, which is the first element that was not in context of the expression in System-T0.
This part of the proof can be reduced to a simple theorem about vectors. 
\sTtoPRSoundSuc
\lookupFromN

The translation of the remaining constructors is either very simple or very technical. We just give the example of the \AgdaInductiveConstructor{App} Constructor.
\begin{itemize}
\item If $f$ is an element of \AgdaDatatype{Exp (n + 1) m} and $x$ is
  an element \AgdaDatatype{Exp 0 m}, then the
  \AgdaInductiveConstructor{App} constructor can be expressed as
  follows. A prime marks translated subexpressions.
	
	$$ (\AgdaInductiveConstructor{App} \ f \ x) \ \rightarrow \ \AgdaInductiveConstructor{C} \ f' \ (\pi_{1}\dots\pi_{n}, \AgdaInductiveConstructor{C} \ x' \ (Z_{n}),\pi_{n + 1}\dots\pi_{n + m}  ) $$ 
\end{itemize}

For the translation of \AgdaInductiveConstructor{App},
\AgdaInductiveConstructor{Lam}, and \AgdaInductiveConstructor{Var} only
\AgdaInductiveConstructor{C} and \AgdaInductiveConstructor{π} are
needed and vice versa. Hence, this theorem is largely independent
of the base type of primitive recursion and can be lifted to all other
instances considered.



%The function that generates the term that mimics the composition is called \AgdaFunction{generalComp}. The following theorem states that the semantic of the returning function is similiar  than the composition operator.
%
%\evalGeneralComp




\section{Conclusions}
\label{sec:conclusions}

\bibliographystyle{jfplike}
\bibliography{jfprefs}

\label{lastpage01}

\end{document}
